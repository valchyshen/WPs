\section{Introduction}

In early May 2024, a conference\footnote{\url{https://www.networkideas.org/announcements/2024/05/sovereign-debt-and-climate-finance-conference/}.} of international development economists was held on the two economic urgencies of the day: sovereign debt and climate finance. Both have been deemed rightfully as crises. However, at the opening panel, featuring top prominent development economists such as Dani Rodrik of Harvard University, a reminder of another urgency arrived during a Q\&A session. At that moment, a development economist representing Ghana\footnote{Charles A. Abugre, a development economist from Ghana, asked the question.} remarked: ``I was struck by all three panelists making absolutely \textit{zero} reference to the changing geopolitical context. Especially war drums beating in all parts of the world -- the war in Europe, the war in the Middle East \dots''~\citep[watch since 12:40 minute, emphasis added]{idea2024}. 

Indeed, at the time, as the conference was held, two signature wars were flaring.\footnote{Alongside of other wars taking place in other countries and harming populations.} In Europe, Russia's war invasion of Ukraine has been progressing for more than 10 years, of which the last two years were of grand escalation.\footnote{Its first stage started in February 2014, when Russia's military invasion resulted in the occupation of two of Ukraine's territories: Crimea and partially Donbas. Russia's leadership launched the second stage of its war on Ukraine on February 24, 2022.} In the Middle East, Israel's war invasion of Palestine in Gaza strip was more than half a year old at the time. Both wars have been devastating to the people under aggression.\footnote{The scale of ruin Ukraine's society has been experiencing is characterized by the following sentence: ``Following the escalation of war on 24 February 2022, the Ukraine refugee crisis became the largest since World War II.''~\citep[p.~2]{unisef2022}.\par Leaders of the countries carrying out the aggressions earned rulings by the International Criminal Court (ICC): in the case of Russia, see \url{https://www.icc-cpi.int/defendant/vladimir-vladimirovich-putin} and \cite{icc2023}, in the case of Israel see \cite{icc2024}.}

This paper assumes this example of the conference's modesty on the present explicitly overwhelming milliary aggressions not as a showcase of omission or neglect but rather of the additional complexity the scholars and experts are facing during the present times. Along with other major goals, this paper aims to evoke such complexity.

State of affairs, when crises are spawning one after another and frequently piling one on another, earned the term of ``polycrisis''~\citep{tooze2022}.\footnote{\citeauthor{tooze2022}, the main popularizer of the term ``polycrisis'' during the tumultuous year of 2022, was hired as contributing editor to \textit{Financial Times}, one of top world's business media, in November 2022~\citep{ft2022a}.} In other inerpretations, it has been termed as ``systemic chaos''~\citep{galanis2024} or ``polycrises''~\citep{dymski2024}. The term polycrisis characterizes the world being grappling with ``overlapping emergencies \dots in which the whole becomes more dangerous than the sum of the parts''~\citep{mckenzie2022}. Currently, under polycrisis it is understood that the whole range of international emergencies since the Covid-19 pandemic in early 2020 and through the present.\footnote{Previous wide usage of the polycrisis term referred to the emergencies during period after the Global Financial Crisis of 2007-09 and through 2016. Namely, the span of the crises were ``the eurozone-Brexit-climate-refugee crises in 2016''~\citep{mckenzie2022}. The first military invasion of Ukraine by Russia in 2014 is omitted, suggesting that the term itself and the scope of its underlying analysis were limited to the perimeter of events that were central to the developed countries. Other emergencies were considered less significant. There are views on polycrisis such as \citep{ft2023,delong2024} that consider the current polycrisis has lasted since 2010.} In addition to the pandemic crisis, the overlapping emergencies include the cost of living crisis (inflation spike of 2021-22) and sovereign debt crises due to the central bank interest rate policy of hiking interest rates. Even the U.S. presidential elections of November 2024, highlighting the electorate's dissatisfaction with incumbent administration, urged immediate observations of (yet another) crisis \citep{mead2024}. Their results, according to an acute observer writing right after the winner was confirmed, fit into just-mentioned conceptualization of chained emergencies: ``The \textit{polycrisis that is unfolding} demands not a return to the status quo but urgent, progressive answers both at home and abroad.''~\citep[emphasis added]{tooze2024}\footnote{This remark was extended in another commentary as: ``\dots I don't think it's an exaggeration to say that for Western Europe, specifically Berlin, Trump is the polycrisis.''~\citep{tooze2024b}} 

The flourishing commentary on the current polycrisis underperforms as it mentions the present war devastations in passing, devoting to them a marginal analysis. It is \textit{much} worse when it muses of those as a showcase of ``the economic and geopolitical agency being exercised by the big middle-income countries''~\citep{mckenzie2022}. Reaching a moral low point, it brands the abovementioned agency as having the ``burn down and rebuild'' strategy of some supposed virture~\citep{sahay2024,sahay2024a}. Meanwhile, these aggressors burn actual families in their homes systematically.\footnote{There is an established track record that describes this aggressive practice as anything but systematic. See \citep{nypost2023,ap2024,cnn2024}.}

This paper shares the perplexity expressed by abovementioned Ghana's economist over the striking relegation of those milliary invasions of grand size and massive humanitarian effect to the far corner of our analysis. Instead, \textbf{[!!! MISSING TEXT !!!]} Such a stance stays ready to explore the most urgent issues, such as sovereign debt, climate finance, and cost of living crises. It would readily look with amusement at the financial markets' supposedly predictive power for gauging the course of solutions of geopolitical issues of the day~\citep{ft2024,tooze2024a}.\footnote{In the runup to November 2024 presidential elections in the U.S., the eurodollar market has been repricing \textit{up} the outstanding sovereign bonds issued by Ukraine's government. Among the money managers, it has been called ``the unlikeliest Trump trade ever''~\citep{ft2024}.} For the peoples of Ukraine and Palestine, the present wars have been a human crisis of extreme proportions that nonresidents cannot feel to the full extent. The following observation appears to be a relevant characterization of the present attitudes: ``It's easier to imagine the end of the world than the end of capitalism.''~\citep{fisher2022} To paraphrase the definition of \citeauthor{fisher2022}'s term ``capitalist realism'': it is easier to imagine a wiping out of yet another Ukraine's town or Palestine's residential block in Gaza than to depart from the habits of thinking and analysis in line with---what \cite{minsky1986} termed as---money manager capialism. This is the latest stage of capitalism we have lived in since the midst of the second half of the twentieth century~\citep{tymoigne2014}.   

Departing from such way of thinking, this paper aims to inquire into international monetary economics. It builds upon the work of American traditional institutionalists such as \citeauthor{veblen1904}, \citeauthor{commons1923}, and \citeauthor{dillard1987} as well as other prominent economists close to that approach such \citeauthor{keynes1936} and \citeauthor{minsky1986}. Also, it adds worthwhile work by less known \citeauthor{innes1913}. 

The point of departure for this paper is the combination of two insights: (i)~\citeauthor{commons1923}'s that one needs to put money at ``the center of economic theory, instead of an afterthought'' in order to get closer to ``the correct picture''~\citep[pp.~644,646]{commons1923}, and (ii)~\citeauthor{dillard1987}'s that one must analyze ``the kind of world in which we actually live''~\citep[p.~1623]{dillard1987}. 

The structure of this paper is the following. The second section revisits a seminal paper `Money as Institution of Capitalism'~\cite{dillard1987} by introducing missing elements such as (i) analysis by \cite{commons1936} that was left out,\footnote{\citeauthor{dillard1987} recognized this ommision at the concluding section of the paper by saying: ``Regretfully some important institutionalists have been left out, such as John R. Commons and Clarence Ayres, both of whom had interesting things to say about money.''~\citep[p.~1644]{dillard1987}} (ii) an explicit primacy of money of account as in \cite{keynes1930a}, and (iii) placing debt-credit relationships at the foundation of the discussion of money as an institution of capitalism as in \cite{innes1913} alongside with money of account. The third section extends the discussion into international terrain.

\section{\cite{dillard1987} `Money as Institution of Capitalism': A Revisit}

On the path of extending the ideas of the \cite{dillard1987} paper into the realm of the present international money manager capitalism, this paper proposes to incorporate the concepts and ideas which were either not stated explicitly, such as money of account, or left out such as contribution of \citeauthor{commons1923}. In addition, this paper introduces \citeauthor{innes1913}, another ``lone non-institutionalist intruder'' as \citeauthor{dillard1987} put it while referencing \citeauthor{keynes1930a}. Lastly, recongition of the international complexity is required, too. In paricular, the dividing line between different economic systems---such as for-profit capitalist versus non-business\footnote{The term `non-business' is borrowed from \cite{dillard1967} where he described the economy of Soviet Union as ``a non-business economy''~\cite[p.~633]{dillard1967}.} socialist---faded for a reasonable degree. 

\subsection{Money of Account as a Primary Concept}

\cite{dillard1987} does \textit{not} use the term `money of account' explicitly. Instead, the author follows \cite{veblen1904} by utilizing the term `money values': ``In business transactions, the bottom line is always expressed in money values.''~\citep[p.~1628]{dillard1987} Saying that transactions are expressed in money values is analogous to saying that transactions are denominated in the money of account.

For analytical purposes of this paper, the concept of money of account is primary. This approach has long tradition, of which \citeauthor{keynes1930a} has been most prominent. \cite{ingham2021} points out ``there is solid evidence that Keynes's conclusion on the \textit{analytical primacy} of money of account came from his study of ancient Near Eastern and Classical Greek metrology and money---his ``Babylonian madness'' during the early 1920s"~\citep[p.~496, emphasis added]{ingham2021}. \citeauthor{keynes1930a} expressed this idea in his two-volume work \textit{Treatise on Money}, where the first paragraph of the very first chapter stated:

\begin{quote}
Money-of-Account, namely that in which Debts and Prices and General Purchasing Power are \textit{expressed}, is the primary concept of a Theory of Money.~\citep[p.~3, emphasis original]{keynes1930a}
\end{quote}

%The expression of debts, prices, and purchasing power in the money of account is primary, while method of recording them---such as informally by word of mouth or formally by book entry---is following. in other words, ``money of account is the \textit{description} or \textit{title},'' while ``the money is [that] which answers to the description''~\citep[pp.~3-4, emphasis original]{keynes1930a}. It is the central authority which defines the name for the money of account for the society. It enforces payment upon debt contracts. Whereas, general economic activities within the society, interactions of its government and private parts determine the values of debts and prices.   

%The descriptive part of this sentence means the money of account has a distinctive and continuously maintained \textit{name}, which might change under cercumstatnces but this is rather rare. Authoriy to name or define the money of account belongs to the state or the central governing body of the community. That is why money of account bear national identity. Some conceptualize money of account as a money unit, which is unit of measurement like inch or meter for length, pound or kilogram for weight, Fahrenheit or Celsius for temperature. However, out of all units of measurement money of account is ``a much more difficult concept''~\citep[p.~145]{wray2020}. This paper follows this conceptualization aiming to contribute to our shared understanding of it.  Hence, examples of money of account are Ghanian cedi, Mexican peso, etc. Respectively, cedi is the money of account of Ghana, peso is of Mexico. Also, the descriptive part has a numerical precision, for example, when a price of some good or service selling in Ghana is expressed as 100 Ghanian cedis, or another good or service selling in Mexico as 100 Mexican pesos. The numeric description standing next to the name of the money of account---one hundred in each case---speaks of money in terms of quantities or values. Some economists speak of these quantities or values as units,\footnote{See discussion of Schumpeter's approach to money in \citep[p.~343, footnote 9]{earley1994} published as part of \cite{dymski1994}.} which is not the path this paper follows. With the money of account as descriptive or title, the name is prerogative of the central authority (government) of the country

%Consider the following real-world example. A resident of Ghana has a 100 Ghanian cedis banknote and a 100 US dollar balance on her checking account in the commercial bank based in Accra, the country's capital. Then, both the Ghanian cedi and the US dollar are money of account. Also, next to them there numerical descriptions---one hundred in earch case---that speak of holder's ability to pay upon the current prices of available goods and services. In other words they are description or title. Then, the cedis banknote and the dollar checking account balance are money which answer to the description.

Yet, \cite{dillard1987} deserves an additonal credit. In particular, it refered, albeit in passing, to an important connection mentioned by \citeauthor{veblen1904} between money of account (named as 'money values') and its broder conceptualization of money as an institution of capitalism. That connection is expressed by observation that ``the money unit enters into the ruling habits of thought of business men,'' the line from \citep[p.~83]{veblen1904} quoted in \citep[p.~1628]{dillard1987}. Here, \citeauthor{veblen1904}'s money unit is identical to \citeauthor{keynes1930a}'s money of account. However, the emphasis on the habit of thought appears more explicit and strong by the former versus the latter.    

%This observation by \citeauthor{veblen1904} with emphasis on the ruling habit of thought is no less important than \citeauthor{keynes1930a}'s primacy of money of account.

%\cite{innes1913} writes on this matter extensively and shows that his own studies predate Keynes'.

The literature highlighting the usage of money of account as a ruling habit of thought include: (i) \citeauthor{colwell1859}'s assertion that the ``use of money of account is a mental operation''~\citep[p.~3]{colwell1859} and that such ``an inveterate mental habit supervenes''~\citep[p.~687]{colwell1860}, (ii) \citeauthor{einaudi1953}'s historical record of the usage of the substitute terms such as `imaginary money' and `ideal money' for `money of account,' acknowledging that ``from time immemorial men have neither seen nor touched any imaginary money''~\cite[p.~230]{einaudi1953},\footnote{It was published as a chapter in \cite{lane1953}.} and (iii) \citeauthor{innes1913} speaking of money of account in similar terms: that ``[t]he eye has never seen'' the money of account since it ``is intangible, immaterial, abstract'' and that people use it as a result of ``long habit'' of thinking in terms of it~\citep[pp.~155,159]{innes1914}.

While \citeauthor{veblen1904} explicily outlined the businessmen as a key group carrying that habit of thinking within the society, for \citeauthor{colwell1859}, \citeauthor{innes1914}, \citeauthor{keynes1930a} and \citeauthor{einaudi1953}~it penetrates entire society albeit with a varying degree. In commerce the abstract, token-free, usage is most profound as being most efficient, while some other parts of the society get accustomed to that habit through the usage of tangable tokens as per~\citep{einaudi1953}. Such a habit, as all others, is developed on the back of an individual's ``past experience, cumulatively wrought out under a given body of traditions, conventionalities, and material circumstances'' and then, importantly, ``[w]hat is true of the individual in this respect is true of the group in which [s]he lives''~\citep[pp.~390-391]{veblen1898}. 

Ultimately, as \cite{keynes1930a} pointed out, the \textit{central} role in defining the money of account and then shaping the entire financial system has been in the hands of the State (its central authority) ``for some four thousand years at least''~\citep[p.~4]{keynes1930a}. It has been doing so through enforcement of debt contracts denominated in the own money of account. Out of myriad of debt contracts within the society or community, the central are tax obligations imposed by State and payment of which it enforces alongside with other debts. \citeauthor{keynes1930a} was explicit in giving credit to \cite{knapp1905} \textit{The State Theory of Money}, or Chartalism. Combinding all said above, the State not only has claimed its right to create and maintain domesic money, it has been creating continiously the material circumstances in which resident individuals reingage with the habit of thought of using \textit{that} money of account. As moeny transacions took place, the wider member of the public shared ``the establishment of habits of this kind'' and it made possible modern banking \citep[p.~23]{keynes1930a}. 

Out of the abovementioned authors, \citeauthor{innes1913} stands out by being not only a proponent of the principles that later became Chartalism in writing, but in praxis too. While working with Egypt finances as part of the Cairo-based British administration, from 1891 and through 1912, he earned reputation of an expert on financial matters. During 1896-1899, he was tasked to serve as a top financial advisor to King of Siam (Thailand). With baht as domesic money of account being already in place, what \citeauthor{innes1913} did first on that job was a revision of the exisiting taxes in Siam, which he put together in a dry report~\citep[see][]{innes1896}.\footnote{It was published in \cite{siam1978}.} Scholars on the Thailand development credited his work directly and indirectly with establishment of a more effecive system of taxation than it was before, see \citep{vella1955,brown1992,queralt2022}.\footnote{\cite{vella1955} does not mention \citeauthor{innes1913} by name while recognizing his input to Siam's modernization drive of the 1890s: ``The most important British adviser was the Financial Adviser engaged in 1896. Since that time the position of the Financial Adviser remained in the British hands.'' \citep[p.~343]{vella1955}.\par Another account observed: ``King Chulalongkorn (r. 1868--1906) was responsible for the giant leap forward in fiscal capacity \dots [as b]etween 1868 and 1915 tax revenue increased almost tenfold, from 8 to 74 million baht" \citep[p.~263]{queralt2022}.} 

\citeauthor{innes1913} is important for the analytical purposes of this paper as his writings \citep{innes1909,innes1910,innes1913,innes1914,innes1914a} help elaborating on the money system as a network of debt-credit relationships denominated in money of account. See next subsecion below.

\subsection{Money as Debt-Credit Relationships}

While \cite{keynes1930a} starts off by effectively saying that debt is money,\footnote{In short, \cite{keynes1930a} explains money as that which discharges debt and then he proceeds by saying ``[a] title to a debt is a title to money at one remove\dots''~\citep[15]{keynes1930a}.\par Today, this presentation prevails not only among the heterodox economists, but is being accepted by a number of mainsream economists \citep[see][]{boe_money}.} the key line from the \citeauthor{innes1913}'s writings states that ``credit and credit alone is money''~\citep[p.~392]{innes1913}. At first glance, it sounds confusing indeed. One can ask how these two authors can stand side by side within the same lineage of thought as, for example, being discussed by~\cite{lakomski}.\footnote{In \cite[p.~490]{lakomski} both Keynes and Mitchell Innes are named as members of the same lineage of economic thought alongside with Schumpeter and few others. Note that in this paper the last name of Mitchell Innes is mentioned as just Innes, a widespread tendency today and among his contemprories who cited his work, of which Keynes was part himself \citep[see][]{keynes1914}.} This subsection aims to moderate the confusion.

It is widely accepted among heterodox economists that money is a \textit{social} relation \citep{ingham1996,lavoie2014}.\footnote{``In the post-Keynesian approach, money is a social relation\dots''~\citep[p.~188]{lavoie2014}. \cite{ingham1996,ingham2004,ingham2020} is more explicit in this conceptualization.} In particular, it is a relation of a debt-credit type~\citep[see][p.~17]{kregel1996}. This said, however, the underlying essense of this type of relationships was not picked up in full from the \citeauthor{innes1913}'s writings by his scholars.\footnote{So far, the most exensive discussion of the \citeauthor{innes1913}'s contribution to monetary theory has been the edited volume \cite{wray2004}. It provides the list of top scholars.} What follows next is a brief exposition of his view.\footnote{A full exposition can be found in \cite{valchyshen2025}.} It is a worhwhile tool for analysis of money as an institution of international capitalism.

The social relationships beween debtors and a creditors, or debt-credit relationships, ``are the most important artificial relations which subsist between human beings.'' They are defined as ``artificial to distinguish them from the [basic human] relations of love and motherhood etc.''~\citep[p.~16]{innes1909} The nature of these relationships has been well realized widely and since quitte long ago:

\begin{quote}
From the merchant of China to the Redskin of America; from the Arab of the desert to the Hottentot of South Africa or the Maori of New Zealand, debts and credits are equally familiar to all, and the breaking of the pledged word, or the refusal to carry out an obligation is held equally disgraceful.~\citep[p.~391]{innes1913}
\end{quote}

It is important to highlight here that debts and credits are mentioned together and the author makes a direct shortcut from them into notion of an obligation or a pledged word. These are the concepts with very close meaining. Indeed, in a pledged word is given by one individual to another. In other words, the former individual has an obligation, promise to the latter. Both of them recognize what exists beween each other: the former individual is to deliver what was agreed, while the latter individual expects the  agreed delivery. In more advanced, or financial, terms: the former individual has an obligation called ``debt \texti{to}'' her counterparty, while the latter individual holds that obligation which is called ``credit \textit{on}'' her counterpary. This is a general description of a debt-credit relationship between two counterparies. In a community consisting of numerous individuals, the way of life is such that an individual at some moment may have accumulated debts to several indiviuals and at the same time have accumulated credits on other individuals.

This social custom, or habit of behavior, which the author named as a ``law'' of debt, ``transcends, perhaps, in importance any other human law''~\citep[p.~16]{innes1909}. However, it is not a money system yet. The historical account of money is out of scope of this paper.\footnote{More so, when ``strach back into the paradisaic inervals of human history of the interglacial periods''~\citep[p.~13]{keynes1930a}.} It just highlights the major building blocks of the \citeauthor{innes1913}'s writings on money. 


\begin{quote}\dots the primitive and the only true commercial or economic meaning of the word ``credit." It is simply the correlative of debt. \dots The words ``credit" and ``debt" express a legal relationship between two parties, and they express the same legal relationship seen from two opposite sides. \citep[p.~392]{innes1913}
\end{quote}

The author expounds an evolutionary process, during which the money system evolved into the one we know of now. As the starting point of that evolutionary process there is a set factors present at the very early stage: (a) there was a central authority within the community within which economic ransacions take places between , and (2) there was a usage of the money of account in which debt-credit relationships are denominated or expressed. 

There were two types of the debt-credit 

\subsection{Evolution of Political and Economic Systems}

As of the year of its publication, the paper dealt with the world that was still clearly divided between capitalist and socialist camps. It assumed the former as nonmonetary economies, a view this paper disagrees with. In just two years, however, that world changed dramatically after the fall of the Berlin Wall in 1989.\footnote{Detailed historical accounts of the fall of the Berlin Wall reveal that money issues played a role at that particular moment. For example, \citeauthor{zubok2021} reconstructs the event in the following terms: ``\dots the popular movement in East Germany produced a spectacularly dramatic moment at the end of October [1989], with mass demonstrations. In view of Gorbachev's refusal to get involved, the younger East German politicians scrambled to act themselves. They sent their aged leaders \dots into retirement, and tried to put down the uprising by promising reforms. The new East German leader Egor Krenz knew that his state was bankrupt: the GDR [German Democratic Republic] had accumulated a large amount of debt that it owed to West Germany. Kranz rushed to Moscow to ask for Soviet assistance, but Gorbachev ignored his appeal: the Soviet Union was running low on foreign currency reserves. Scrambling for solutions, Krenz and his comrades East German citizens state-regulated travel to West Berlin. In the midst of their chaotic moves, an error by one confused official led to an unexpected release of pent-up tension: the opening of the Berlin Wall. On the night of 9 November 1989, a confused border guard let a jubilant and stunned crowd of East Germans pass through formidable checkpoints and pour into West Berlin.''~\citep[p.~93]{zubok2021}\par The overarching money issue at the time was captured by former chancellor of Germany Helmut Schmidt who once asserted that ``monetary policy is foreign policy''~\citep[quoted in][p.~49]{blackwill}.} The Soviet Union collapsed and vanished altogether in 1991, in just another two years~\citep{zubok2021}. 

Moreover, implicitly \cite{dillard1987} assumes the 

by extending it to the international terrains. It follows his advice by analyzing ``the kind of world in which we actually live.''~\citep[p.~1623]{dillard1987} It changed to an extent. As of \citeyear{dillard1987}, when \citeauthor{dillard1987} wrote that paper, that kind of world was still clearly divided between capitalist and socialist camps. Just in two years, the former camp started vanishing, and in just another two years, that process was complete. Few countries under a single communist party rule did survive. However, the prevailing underlying principle of operations of the emerging kind of world since the 1990s was capitalism. If in the most advanced capitalist countries ``the dominant institution and idea [remained that of was] of capital'' as per \citep[p.~40]{ayres1944}, in many countries turned to the market economy model those were operational if not dominant in outright way. 

\citep{rudd2024}

One of the main goals of this paper is to explain money as an institution of international capitalism.

\citep[pp.~382-383]{veblen1898}: ``Of a
 similar import is the characterization of money as ``the
 great wheel of circulation'' t or as ``the medium of exchange.''  Money is here discussed in terms of the end
 which, " in the normal case," it should work out according
 to the given writer's ideal of economic life, rather than in
 terms of causal relation.
 With later writers especially, this terminology is no
 doubt to be commonly taken as a convenient use of meta
 phor, in which the concept of normality and propensity to
 an end has reached an extreme attenuation. But it is pre
 cisely in this use of figurative terms for the formulation of
 theory that the classical normality still lives its attenuated
 life in modern economics; and it is this facile recourse to
 inscrutable figures of speech as the ultimate terms of the
 ory that has saved the economists from being dragooned
 into the ranks of modern science. The metaphors are
 effective, both in their homiletical use and as a labor-sav
 ing device,- more effective than their user designs them
 to be. By their use the theorist is enabled serenely to
 enjoin himself from following out an elusive train of causal
 sequence. He is also enabled, without misgivings, to con
 struct a theory of such an institution as money or wages or
 land-ownership without descending to a consideration of
 the living items concerned, except for convenient corrobo
 ration of his normalized scheme of symptoms.''


\section{Money as Institution of Capitalism: Adding International Dymansion}

``"Capitalism" is that economy of which the dominant institution and idea is that of capital." \citep[p.~40]{ayres1944}
